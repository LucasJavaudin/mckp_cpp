\documentclass{article}

\usepackage[utf8]{inputenc} 
\usepackage[T1]{fontenc}
\usepackage[french]{babel}
\usepackage[margin=2.5cm]{geometry}
\usepackage{hyperref}

\title{Projet C++ : problème du sac à dos à choix multiple et politique d'incitations}
\author{\textsc{Javaudin} Lucas, \textsc{Le Rest} François}

\begin{document}

\maketitle

\tableofcontents

\newpage

\section{Introduction}

En recherche opérationnelle, le problème du sac à dos\footnote{Voir \url{en.wikipedia.org/wiki/Knapsack_problem}} est un problème d'optimisation dont l'objectif est de maximiser la valeur des objets que l'on insère dans un sac qui ne peut pas supporter plus d'un certain poids.

Nous étudions ici une variante du problème du sac à dos : le problème du sac à dos à choix multiple ou \textit{multiple choice knapsack problem} (MCKP).
Dans le MCKP, les objets sont regroupés en plusieurs classes et le sac à dos doit être rempli avec un et un seul objet de chaque classe.

Le MCKP est un problème NP-complet.
Nous proposons plusieurs algorithmes qui donnent une approximation de la solution optimale.
Plus précisément, les solutions proposées par ces algorithmes sont réalisables et leur distance par rapport à la solution optimale est bornée.

Enfin, nous montrons que le problème du sac à dos à choix multiple est équivalent au problème consistant à trouver la politique d'incitations maximisant l'utilité sociale, lorsque les individus font face à un choix discret et lorsque le budget de la politique est contraint.

\section{Algorithmes}
% On peut mettre ici un pseudo-code des algorithmes que l'on a codé.

\subsection{Recherche exhaustive}

\subsection{Algorithme Greedy simple}

\subsection{Algorithme Greedy avec enveloppe convexe}

\subsection{Algorithme Dyer-Zemel}

\section{Structure du code}
% On peut mettre ici un diagramme de classe UML et une présentation des fichiers avec leur contenu.

\section{Application économique}
% Je vais ici parler de l'application du MCKP à une politique d'incitation.

\section{Références}

\begin{thebibliography}{9}
\bibitem{Knapsack2004}
  Hans Kellerer, Ulrich Pferschy and David Pisinger,
  \textit{Knapsack problems},
  Springer,
  2004.
\end{thebibliography}

\end{document}
